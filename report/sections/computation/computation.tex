\section{Computational Details}
In this section we will go through some of more details on how to set up the problem in order to solve it.  
\subsection{Operator splitting}
The meat of what we are looking to solve is the equation
\begin{equation}
 {\partial v \over \partial t} = \nabla \cdot (M_i \nabla) - I_{ion} (v,s)
\end{equation}
we could in principle solve this in one round, but as we do not have complete information about $I_{ion}(v,s)$, we typically can only rely on goss to solve the equation 
\begin{equation}
 {\partial v \over \partial t} = -I_{ion}(v,s)
\end{equation}
for one time step. Fortunately, this is exactly what we need when we use the technique called operator splitting. The basic idea is the following: consider a problem of the form 
\begin{align}
 {\partial v \over \partial t} &= (L_1 + L_2)v \\
 v(0) &= v_0
\end{align}
We can then find a trial solution by first solving 
\begin{align}
 {dv\over dt} &= L_1(u) \\
 u(0) = v_0
\end{align}
to $\Delta t$, then solve 
\begin{align}
 {dw \over dt} = L_2(w)
 w(0)  = u(\Delta t)
\end{align}
and then use $w(\Delta t)$ as an approximation to $v(\Delta t)$. Our code uses a slighty more advanced method, and both methods are discussed thoroughly in \cite{Sundnes06}, p. 71-82.

\subsection{Solving the PDE part by finite elements}
Is it a waste of time to write this? prolly. 